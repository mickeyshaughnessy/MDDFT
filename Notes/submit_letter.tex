\documentclass[10pt]{letter}
\usepackage{epsfig}
\begin{document}
\pagestyle{plain}

\signature{Reese Jones and Mickey Shaughnessy}
\begin{letter} {}


\begin{minipage}[h] {0.45\textwidth}
\setlength{\unitlength}{\textwidth}
\begin{picture} (1,0.01)
\put(0,0){\line(1,0){1}}
\end{picture}
\vskip 0.1in
\epsfig{file=SandiaBird.eps,width=\textwidth}
\begin{picture} (1,0.01)
\put(0,0){\line(1,0){1}}
\end{picture}
\protect\vspace{0.15in}
\end{minipage}\hfill%
\begin{minipage}[h] {0.45\textwidth}
Reese Jones \\
P. O. Box 969, \\
Sandia National Laboratories, \\
Livermore, CA 94551-0969, USA \\
(925) 294-4744 \\
rjones@sandia.gov

\protect\vspace{0.1in}
\end{minipage}

\begin{minipage}[h] {0.45\textwidth}
The Journal of Chemical Theory and Computation\\
\end{minipage}

%\protect\vspace{-0.5in}

\opening{Dear Editors,}


%DATE

%Dear Editors,

We would like to submit our manuscript 
``Efficient use of {\it ab initio} calculations to generate accurate Newtonian dynamics ''
for publication in the {\it The Journal of Chemical Theory and Computation} as a article.
In the paper we develop a new method of using {\it ab initio} calculations to drive classical dynamics that has the potential to scale to system sizes approaching those feasible with traditional empirical molecular dynamics. 
The method relies on a database of pre-computed Hellmann-Feynman forces associated with local atomic configurations. 
Using a distance metric on these configurations we are able to do fast searches through the database for similar configurations and also to form a local interpolation of the stored forces at configurations generated by the dynamics.
This pairing of configurations and forces supplants the traditional globally fitted empirical potential. 
The approach is shown to be convergent and  have acceptable accuracy and conservation properties.
All details necessary to apply the method are given.

\closing{Sincerely,}
\end{letter}
\end{document}

%Sincerely yours,

%CORRESPONDING AUTHOR

%AUTHOR'S PHONE NUMBER
%AUTHOR'S FAX NUMBER
%AUTHOR'S EMAIL

%AUTHOR'S POSTAL ADDRESS


COLOR FIGURES

Justify color on figures (or indicate that you are
willing to pay for the reproduction of nonessential color).
Nonessential color will be reproduced in monochrome in the
print journal, color in the online journal.  Authors *must*
justify color.  A bald statement that color is essential will
not suffice.

Color is "essential" if its use compared to monochrome allows the
author to present a significant amount of additional, important
information to the reader, or if it makes it significantly easier
for the reader to interpret and understand the important information
in the figure.

FURTHER INFORMATION (tell us if this work contradicts the
published work of others who may want to see the paper before 
it is published, etc.)

